\pdfminorversion=4
\documentclass[aspectratio=169]{beamer}

\mode<presentation>
{
  \usetheme{default}
  \usecolortheme{default}
  \usefonttheme{default}
  \setbeamertemplate{navigation symbols}{}
  \setbeamertemplate{caption}[numbered]
  \setbeamertemplate{footline}[frame number]  % or "page number"
  \setbeamercolor{frametitle}{fg=white}
  \setbeamercolor{footline}{fg=black}
} 

\usepackage[english]{babel}
\usepackage[utf8x]{inputenc}
\usepackage{tikz}
\usepackage{courier}
\usepackage{array}
\usepackage{bold-extra}
\usepackage{minted}
\usepackage[thicklines]{cancel}
\usepackage{fancyvrb}

\xdefinecolor{dianablue}{rgb}{0.18,0.24,0.31}
\xdefinecolor{darkblue}{rgb}{0.1,0.1,0.7}
\xdefinecolor{darkgreen}{rgb}{0,0.5,0}
\xdefinecolor{darkgrey}{rgb}{0.35,0.35,0.35}
\xdefinecolor{darkorange}{rgb}{0.8,0.5,0}
\xdefinecolor{darkred}{rgb}{0.7,0,0}
\definecolor{darkgreen}{rgb}{0,0.6,0}
\definecolor{mauve}{rgb}{0.58,0,0.82}

\title[2018-09-07-uchicago-what-am-i-doing]{\huge What am I doing?}
\author{Jim Pivarski}
\institute{Princeton University -- DIANA-HEP}
\date{September 7, 2018}

\usetikzlibrary{shapes.callouts}

\begin{document}

\logo{\pgfputat{\pgfxy(0.11, 7.4)}{\pgfbox[right,base]{\tikz{\filldraw[fill=dianablue, draw=none] (0 cm, 0 cm) rectangle (50 cm, 1 cm);}\mbox{\hspace{-8 cm}\includegraphics[height=1 cm]{princeton-logo-long.png}\includegraphics[height=1 cm]{diana-hep-logo-long.png}}}}}

\begin{frame}
  \titlepage
\end{frame}

\logo{\pgfputat{\pgfxy(0.11, 7.4)}{\pgfbox[right,base]{\tikz{\filldraw[fill=dianablue, draw=none] (0 cm, 0 cm) rectangle (50 cm, 1 cm);}\mbox{\hspace{-8 cm}\includegraphics[height=1 cm]{princeton-logo.png}\includegraphics[height=1 cm]{diana-hep-logo.png}}}}}

% Uncomment these lines for an automatically generated outline.
%\begin{frame}{Outline}
%  \tableofcontents
%\end{frame}

% START START START START START START START START START START START START START

\begin{frame}{}
\Large
\vspace{0.5 cm}
\begin{center}
\textcolor{darkblue}{It's good to give a talk like this because I don't have any \\ ``status report'' meetings to force myself to review.}
\end{center}
\end{frame}

\begin{frame}{End goal: query service}
\large
\vspace{0.5 cm}
I often talk about a ``query service'' to end our reliance on private skims before they become infeasible (HL-LHC).

\begin{center}
\includegraphics[width=0.5\linewidth]{basic-block-diagram.pdf}
\end{center}

\uncover<2->{\textcolor{darkblue}{That's still my end goal, which I use to judge the value of each of my subprojects, but what I'm actually doing is several steps removed from that.}}
\end{frame}

\begin{frame}{Pieces of a query service}
\large
\vspace{0.5 cm}
\begin{itemize}
\item \textcolor{darkblue}{Distributed task scheduling:}

\item \textcolor{darkblue}{Query language:}

\item \textcolor{darkblue}{Columnar data storage/management:}

\item \textcolor{darkblue}{Columnar ROOT access:}

\item \textcolor{darkblue}{Manipulating event/subevent structure:}

\end{itemize}
\end{frame}




\begin{frame}{uproot is crazy popular}
\vspace{0.5 cm}
Nearly 1~year from version~1.0 to version~3.0.0b2 $\to$ 551 unique downloaders.

\textcolor{gray}{\scriptsize (``Uniqueness'' defined as same country code, OS distribution, version, and kernel; may over- or under-count.)}

\includegraphics[height=4.2 cm]{uproot-pip-vsweek.pdf}
\only<1>{\hfill \includegraphics[height=4.2 cm]{uproot-pip-numdownloads.pdf}}
\only<2-3>{\includegraphics[height=4.2 cm]{uproot-pip-vscountry.pdf}}
\only<2>{\includegraphics[height=4.2 cm]{uproot-pip-vsos.pdf}}
\only<3>{\includegraphics[height=4.2 cm]{uproot-pip-vsoscountry.pdf}}
\end{frame}



\end{document}
